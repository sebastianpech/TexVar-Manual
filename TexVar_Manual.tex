\documentclass[fleqn]{mybase}
\usepackage{mysimple}
\usepackage{texvar}
\usepackage{epstopdf}
\usepackage{mdframed}
\graphicspath{{Abbildungen/}}

\title{TexVar - Manual}
\subtitle{TexVar - LaTeX math calculations\\[0.5 cm]\normalsize\directlua{tVar.getVersion()}}
\date{\today}
\author{Sebastian Pech}

\addbibresource{TexVar_manual.bib}
% Create new environment for tVar Output
\mdfdefinestyle{tVarSTYLE}
{
linewidth=1pt,
innerleftmargin=0.5cm,
innertopmargin=0cm,
userdefinedwidth=13cm,
align=center,
frametitle={\hfill Output},
rightline=false,
leftline=false,
backgroundcolor=gray!10,
needspace=20em
}

%\begin{luacode*}
%function tVar.units:printStackTex(layer)
%	layer = layer or 0
%	if layer == 0 then
%		tex.print("\\si{" .. (self.identifier or " ").. "}")
%	end
%	tex.print("\\begin{itemize}[nosep,noitemsep]")
%	for i,v in self:iterateStack() do
%		local ident = ""
%		if v.unit.identifier ~= nil then 
%			ident = v.unit.identifier
%		end
%		tex.print("\\item " .. v.operation .. "\\," .. v.factor .. "\\," .. "\\si{" .. ident .. "}" ..  "")
%		if not v.unit:isLeaf() then
%			v.unit:printStackTex(layer+1)
%		end
%	end
%	tex.print("\\end{itemize}")
%end
%\end{luacode*}

\newenvironment{tVarOUT}
{
\begin{mdframed}[style=tVarSTYLE]
}
{
\end{mdframed}
}

\begin{document}
\selectlanguage{english}
\maketitle
\tableofcontents
\section{Description}
TexVar (short tVar) is a basic \LaTeX~math calculations tool written in Lua.
For integration into \LaTeX~it has to be used together with LuaLaTeX. Compared
to software like Mathcad TexVar is a lot more flexible. You can fill custom
designed tables with results, do calculations within text documents and print
beautiful LaTeX equations. The current version also supports 2D-plotting with
gnuplot.
\section{Installation}
\subsection{Prerequisites}
The following software is needed in order to use TexVar:
\begin{itemize}
	\item Lua 5.1 or higher
	\item LuaLaTeX (MikTex or Texlive)
	\item GnuPlot 5.0 (only needed for plotting)
\end{itemize}
By default the command \verb|\si{}| is used for displaying units, so the
siunitx package is mandatory. If you want to define your own command you can
change the value of \verb|tVar.unitCommand = "\\si"|. 
\subsection{Installation}
Download TexVar from \url{https://gitlab.com/spech/TexVar} and copy the folder
\emph{tVar} and the file \emph{texvar.sty} (subfolder package) to a location
that is visible to \LaTeX. (e.g. the folder your *.tex file is in or a global
\LaTeX-folder \footnote{You can find information on global \LaTeX-folders
at\url{https://www.math.hmc.edu/computing/support/tex/installing/}})
\subsubsection{Configuration for Gnuplot}
In order to use Gnuplot with TexVar you have to allow LuaLaTeX to call external
commands during runtime. This works through the command-line switch
\lstinline[language=Bash]|--shell-escape|. Your complete call for LuaLaTeX
could look like this: \lstinline[language=Bash]|lualatex -synctex=1
-interaction=nonstopmode --shell-escape %.tex|. 
\section{Getting Started - Hello World}
To ensure your installation is working test the following code. When using an
luacode based environment like tVar it's important that the commands
\lstinline|\begin{tVar} \end{tVar}| are \emph{not indented}.
\begin{lstlisting}
\documentclass{article}
%
\usepackage{texvar}
\usepackage[fleqn]{amsmath}
%
\begin{document}
\begin{tVar}
	#Hello World! I'm using TexVar
	tVar.getVersion()
\end{tVar}
\end{document}
\end{lstlisting}
\begin{tVarOUT}
\begin{tVar}
	tVar.logInterp = true 
	#Hello World! I'm using TexVar
	tVar.getVersion()
\end{tVar}
\end{tVarOUT}
\section{General Information on tVar Environment and Macro}
To execute TexVar commands from \LaTeX, the TexVar package offers two options:
\begin{itemize}
	\item \lstinline|\begin{tVar} \end{tVar}|, for commands blocks
	\item \lstinline|\tv{}|, for inline calls
\end{itemize}
Both forward the commands to the TexVar interpreter.
\begin{lstlisting}
\begin{tVar}
	a:=10
	b:=3
	c:=a+b
\end{tVar}
The result is: \tv{c:outRES()}
\end{lstlisting}
\begin{tVarOUT}
\begin{tVar}
{
a:=10
b:=3
c:=a+b
}
\end{tVar}
The result is: \tv{c:outRES()}
\end{tVarOUT}
Generally, there are two key commands the TexVar interpreter searches for
\footnote{Details on these key commands are discussed in Section
\ref{lbl:tvarcommands}}:
\begin{enumerate}
	\item \#, for text output
	\item :=, for calculations and assignments
\end{enumerate}
Every other command is directly forwarded to the Lua interpreter. This means
you can write any Lua code you want inside the tVar environment. 
\begin{lstlisting}
\begin{tVar}
	function mypow(a,b)
		return a^b
	end
	for i=0,2 do
		tex.print("Step " .. mypow(2,i) .. "\\\\")
	end
\end{tVar}
\end{lstlisting}
\begin{tVarOUT}
\begin{tVar}
	function mypow(a,b)
		return a^b
	end
	for i=0,2 do
		tex.print("Step " .. mypow(2,i) .. "\\\\")
	end
\end{tVar}
\end{tVarOUT}
\section{TexVar - Commands}
\label{lbl:tvarcommands}
\subsection{Creating Variables}
\label{lbl:creatingVars}
TexVar knows three types of variables: 
\begin{labeling}{xxxxx}
	\item[tVar] is the basic type for scalar values.
	\item[tMat] is used for matrices.
	\item[tVec] is used for vectors.
\end{labeling}
The following code generates a variable of each type. Every assignment is made
with $:=$.
\begin{lstlisting}
\begin{tVar}
	a:=13
	e:={1,2,4}
	A:={{1,0},{2,4},{3,9.1}}
\end{tVar}
\end{lstlisting}
\begin{tVarOUT}
\begin{tVar}
a:=13
e:={1,2,4}
A:={{1,0},{2,4},{3,9.1}}
\end{tVar}
\end{tVarOUT}
\subsubsection{Creating Variables in Equations}
New variables can also be created inside equations. The notation for vectors
and matrices is explained in Section \ref{lbl:creatingVars}. Lua knows - by
default - how to do calculations with numbers, so any equation which does not
contain tVar objects, can only be printed as it's result. Any mathematical
operation with a tVar object and a decimal number results in a tVar value. 
\begin{lstlisting}
\begin{tVar}
	# Variable a is created from the equation ...
	a:=13*2^2
	# To preserver the equation some of the ...
	a:=tVar:New(13)*2^2
	# At first Lua calculates $2^2=4$ ...
	# To print the whole ...
	a:=13*tVar:New(2)^2
	# Matrices and Vectors
	c:={{1,2,3},{4,2,6},{1,3,2}}*{1,2,3}*a
\end{tVar}
\end{lstlisting}
\begin{tVarOUT}
\begin{tVar}
# Variable a is created from the equation $13\\cdot2^2$. Since all numbers are just decimal values, Lua calculates the result and creates the tVar object afterwards.
a:=13*2^2
# To preserver the equation some of the numbers have to be initialized as tVar objects. In this case it's the number 13.
a:=tVar:New(13)*2^2
# At first Lua calculates $2^2=4$ and then multiplies the tVar object 13 with $4$ which results in a tVar object containing the equation.
# To print the whole equation, one number from the first calculation has to be transformed into a tVar object. 
a:=13*tVar:New(2)^2
# Matrices and Vectors
c:={{1,2,3},{4,2,6},{1,3,2}}*{1,2,3}*a
\end{tVar}
\end{tVarOUT}
\subsubsection{Auto-Formatting of variable names}
\label{lbl:autoformatting}
The \LaTeX-representation of a variable is automatically generated from the
variable name. The first occurrence of an underline starts the subscript the
first occurrence of a double underline starts the superscript. Every other
underline becomes a comma.
\begin{lstlisting}
\begin{tVar}
	a_1_3:=13
	a__10_2:=3
	b_1_x__y_2:=12
\end{tVar}
\end{lstlisting}
\begin{tVarOUT}
\begin{tVar}
{
	a_1_3:=13
	a__10_2:=3
	b_1_x__y_2:=12
}
\end{tVar}
\end{tVarOUT}
\subsection{Creating Functions}
\label{lbl:creatFun}
Functions can easily be defined with the following syntax. The auto-formatting
of function names and attribute names works according to Section
\ref{lbl:autoformatting}.
\begin{lstlisting}
\begin{tVar}
	f(x,y):=x^2+y^2+4
	a:=f(2,3)+11
\end{tVar}
\end{lstlisting}
\begin{tVarOUT}
\begin{tVar}
f(x,y):=x^2+y^2+4
a:=f(2,3)+11
\end{tVar}
\end{tVarOUT}
\subsection{Indexing Matrices and Vectors}
The syntax for indexing a matrix is similar to the Matlab syntax. To access a
matrix you have to use square brackets and a string as key. The key has to be
formatted according to the following examples.
\begin{lstlisting}
\begin{tVar}
	A:={{1,2,6},{2,4,6},{7,6,9}}
	# Index one element with syntax [row,column]
	A["1,2"]:outRES()
	# Index a range 
	A["1:2,1:end"]:outRES()
	# The range 1:end is equal to :
	A["1:2,:"]:outRES()
	# You can also set matrices this way
	A["1:2,:"]:={{1,2,3},{4,5,6}}
	A["1:2,:"]:outRES()
	# Adress vector
	c:=A[":,2"]
	d:=A["2,:"]
	# The above also applies to vectors
	{plain
		v_1:={1,4,3}
		#,~
		v_1["2"]:outRES()
		#,~
		v_1["1"]:=9
		v_1:outRES()
	}
\end{tVar}
\end{lstlisting}
\begin{tVarOUT}
\begin{tVar}
	A:={{1,2,6},{2,4,6},{7,6,9}}
	# Index one element with syntax [row,column]
	A["1,2"]:outRES()
	# Index a range 
	A["1:2,1:end"]:outRES()
	# The range 1:end is equal to :
	A["1:2,:"]:outRES()
	# You can also set matrices this way
	A["1:2,:"]:={{1,2,3},{4,5,6}}
	A["1:2,:"]:outRES()
	# Adress vector
	c:=A[":,2"]
	d:=A["2,:"]
	# The above also applies to vectors
{plain
	v_1:={1,4,3}
	#,~
	v_1["2"]:outRES()
	#,~
	v_1["1"]:=9
	v_1:outRES()
}
\end{tVar}
\end{tVarOUT}
The command \lstinline|{plain }| groups equations to one math environment. For
details see Section \ref{sec:groups}
\subsection{Output}
\subsubsection{print()}
\label{sec:OUTPUT_PRINT}
The \lstinline|[tVar]:print()| command creates the output according to the
global parameter \lstinline|tVar.OUTPUT_MODE|. The following options are
supported:\\
\begin{tVarOUT}
\begin{tVar}
{
	a:=10
	b:=2
	c:=(a+b):print()
}
	#\\verb|tVar.OUTPUT_MODE = \"RES_EQ_N\"|
	tVar.OUTPUT_MODE = "RES_EQ_N"
	c:print()
	#\\verb|tVar.OUTPUT_MODE = \"RES_EQ\"|
	tVar.OUTPUT_MODE = "RES_EQ"
	c:print()
	#\\verb|tVar.OUTPUT_MODE = \"RES\"|
	tVar.OUTPUT_MODE = "RES"
	c:print()
\end{tVar}
\end{tVarOUT}
If you do your calculations inside \lstinline|\begin{tVar}| and
\lstinline|\end{tVar}| the \lstinline|[tVar]:print()| command is automatically
added to your calculations. This means \lstinline|c:=(a+b)| creates an output
using the \lstinline|[tVar]:print()| command. To suppress the automatic output
just add a ";" at the end of the line.
\begin{lstlisting}
\begin{tVar}
	# With output:
	a:=10
	b:=3
	c:=a+b
	# Without output (Lines 7 and 8 produce no output):
	a:=10;
	b:=3;
	c:=a+b
\end{tVar}
\end{lstlisting}
\begin{tVarOUT}
\begin{tVar}
		tVar.OUTPUT_MODE = "RES_EQ_N"
# With output:
{
a:=10
b:=3
c:=a+b
}
# Without output (Lines 7 and 8 produce no output):
a:=10;
b:=3;
c:=a+b
\end{tVar}
\end{tVarOUT}
\subsubsection{outRES()}
The \lstinline|[tVar]:outRES()| command is equal to the combination of
\lstinline|[tVar]:print()| with \lstinline|tVar.OUTPUT\_MODE = "RES"|. If you
do your calculations inside \lstinline|\begin{tVar}| and
\lstinline|\end{tVar}| the \lstinline|[tVar]:outRES()| command is automatically
added to any assignment of a new variable. That means \lstinline|a:=10| creates
and output using the \lstinline|[tVar]:outRES()| command.

In some cases you might want to print an equation without assigning it to a
variable. This can be achieved by directly calling the function
\lstinline|tVar.outRES([tVar])| (now with a dot) and passing your equation as
argument to the function.
\begin{lstlisting}
\begin{tVar}
	a:=2
	b:=13
	tVar.outRES((a+b)/2+b)
\end{tVar}
\end{lstlisting}
\begin{tVarOUT}
\begin{tVar}
	{
	a:=2
	b:=13
}
	tVar.outRES((a+b)/2+b)
\end{tVar}
\end{tVarOUT}
This method also works for the output commands listed in Section
\ref{sec:outputcommands}.
\subsubsection{Other Output Commands}
\label{sec:outputcommands}
For every output mode mentioned in Section \ref{sec:OUTPUT_PRINT} there is an
equal output function.
\begin{labeling}{RESxEQxxN}
	\item[RES\_EQ\_N] \lstinline|[tVar]:outRES_EQ_N(numbering,environment)|
	\item[RES\_EQ] \lstinline|[tVar]:outRES_EQ(numbering,environment)|
	\item[RES] \lstinline|[tVar]:outRES(numbering,environment)|
\end{labeling}
The attributes \emph{numbering} and \emph{environment} are boolean values and
define if the output is created with numeration and a math environment. Both
parameters are optional and can also be defined via the global parameters
\lstinline|tVar.EQUATION_NUMBERING| and
\lstinline|tVar.MATH_ENVIRONMENT|\footnote{In case
\lstinline|tVar.MATH_ENVIRONMENT| is set to "" no math environment is used.}

Additionally there are the functions  \lstinline|[tVar]:out()| for printing the
result without the variable name and \lstinline|[tVar]:outN()| for printing the
variable name the equation with numbers and the result.

\subsubsection{\LaTeX~Output}
Inside \lstinline|\begin{tVar}| and  \lstinline|\end{tVar}| the symbol \# can
be used to print a text in \LaTeX. Inside such a text you can use
\%\%[tVar]\%\% to print a variables name and \$\$[tVar]\$\$ to print a
variables value.

\LaTeX~commands used within \# have to be escaped. For example
\lstinline|\textbf{}| has to be written as \lstinline|\\textbf{}|. 
\begin{lstlisting}
\begin{tVar}
	#\\subsection*{First Subsection}
	a_1_2:=22.4:setUnit("m")
	# Linebreaks are a bit confusing \\\\
	# The variable %%a_1_2%% has the value $$a_1_2$$	
\end{tVar}
\end{lstlisting}
\begin{tVarOUT}
\begin{tVar}
#\\subsection*{First Subsection}
a_1_2:=22.4:setUnit("m") 
# Linebreaks are a bit confusing \\\\
# The variable %%a_1_2%% has the value $$a_1_2$$	
\end{tVar}
\end{tVarOUT}
\subsubsection{Precise Manipulation of Output}
The following functions manipulate equations directly and can be used for
detailed formatting.
\begin{lstlisting}
\begin{tVar}
	a:=10:setUnit("m")
	b:=3:setUnit("m")
	c:=((((a+b):bracR()^2):bracB()*5):bracC()*22):setUnit("m^2")
	#In case you do a really long calculation you can insert a linebreak\\\\
	c:=((a+a+a+a+a+a+a+a+b+b):CRLF("+")+b+b+b:CRLF_EQ("+")+b+b+b+a+a+a+a):CRLFb("=")
	tVar.DEBUG_MODE = "off"
\end{tVar}
\end{lstlisting}
\begin{tVarOUT}
\begin{tVar}
	a:=10:setUnit("m")
	b:=3:setUnit("m")
	c:=((((a+b):bracR()^2):bracB()*5):bracC()*22):setUnit("m^2")
	#In case you do a really long calculation you can insert a linebreak\\\\
	c:=((a+a+a+a+a+a+a+a+b+b):CRLF("+")+b+b+b:CRLF_EQ("+")+b+b+b+a+a+a+a):CRLFb("=")
	tVar.DEBUG_MODE = "off"
\end{tVar}
\end{tVarOUT}
\begin{table}[ht]\centering
	\begin{tabular}{@{}p{5.6cm}p{8cm}@{}}\toprule
		\lstinline|[tVar]:bracR()| & Surrounds the [tVar] object with round brackets. \\	
		\lstinline|[tVar]:bracB()| & \dots~ boxed brackets. \\
		\lstinline|[tVar]:bracC()| & \dots~ curly brackets. \\
		\midrule
		\lstinline|[tVar]:bracR_EQ()| & Surrounds the equation part of the [tVar] object with round brackets. \\	
		\lstinline|[tVar]:bracB_EQ()| & \dots~ boxed brackets. \\
		\lstinline|[tVar]:bracC_EQ()| & \dots~ curly brackets. \\
		\midrule
		\lstinline|[tVar]:bracR_N()| & Surrounds the numerical part of the [tVar] object with round brackets. \\	
		\lstinline|[tVar]:bracB_N()| & \dots~ boxed brackets. \\
		\lstinline|[tVar]:bracC_N()| & \dots~ curly brackets. \\	
		\midrule
		\lstinline|[tVar]:CRLF ([string])| & Inserts a line break in N after the [tVar] object and adds the string before and after the line break. \\
		\lstinline|[tVar]:CRLFb([string])| & Inserts a line break in N before the [tVar] object and adds the string before and after the line break. \\
		\lstinline|[tVar]:CRLF_EQ ([string])| & Inserts a line break in EQ after the [tVar] object and adds the string before and after the line break. \\
		\lstinline|[tVar]:CRLFb_EQ([string])| & Inserts a line break in EQ before the [tVar] object and adds the string before and after the line break. \\
		\lstinline|[tVar]:clean()| & Removes the calculation history from an object.\\
		\lstinline|[tVar]:setUnit([string])| & Sets the unit for a tVar object.\\
		\lstinline|[tVar]:setFormat([string])| & Sets the numberformat for a tVar Object. For details see Section \ref{sec:numberformat}\\
		\bottomrule\end{tabular}
\end{table}
\subsubsection{Number Format \label{sec:numberformat}}
The number format can be controlled globally and locally. The function
\lstinline|[tVar]:setFormat([string])| defines the local number format of a
tVar object. In case no local format was defined, tVar falls back to the global
format set with \lstinline|tVar.NUMBERFORMAT = [string]|. 

Common values are:
\begin{tVarOUT}
\begin{tVar}
	# Variable a is set to $192.6345$\\\\
	# Integer \\lstinline|%d|
	a:=192.6345:setFormat("%d")
	# Exponential \\lstinline|%.2E|
	a:=192.6345:setFormat("%.2E")
	# Float \\lstinline|%.3f|
	a:=192.6345:setFormat("%.3f")

\end{tVar}
\end{tVarOUT}
There is one case where TexVar automatically changes the number format: When
the output precision return $0$ but the calculation precision is unequal to $0$
the number format is changed to \lstinline| %.3E|.
\begin{tVarOUT}
\begin{tVar}
	# Variable a is set to $0.00001$. Default number format is \\lstinline|%.3f|. Calculation precision is $10$
		a:=0.00001
		b:=123432
		c:=a*b
\end{tVar}
\end{tVarOUT}
\subsubsection{Grouping Math Environments}
\label{sec:groups}
By default TexVar creates a new mathematical environment for every output
(except plain text-output with \#). To group equations into one environment,
you can enclose them with curly brackets. The group operator automatically adds
an alignment symbol at the beginning of the line and a line-break at the end.
If you want to suppress this behavior you can open the group with the command
\lstinline|{plain|.
\begin{lstlisting}
\begin{tVar}
	{
		a:=1
		b:=10
		c:=3
	}
	d:=(a+b):bracR()/c
	#It's also possible to create environments without line-breaks and alignment:
	{plain
		f:=3
		#,~
		g:=f+d
	}
\end{tVar}
\end{lstlisting}
\begin{tVarOUT}
\begin{tVar}
{
	a:=1
	b:=10
	c:=3
}
	d:=(a+b):bracR()/c
#It's also possible to create environments without line-breaks and alignment:
{plain
	f:=3
	#,~
	g:=f+d
}
\end{tVar}
\end{tVarOUT}
\subsection{Global Parameters}
\label{sec:globparam}
Global parameters can be set during runtime and affect all commands.

\begin{table}[H]\centering
	\captionabove{Global parameters with default values and description}
	\label{tab:globalparam}
	\begin{tabular}{@{}lp{8cm}@{}}\toprule
		\lstinline|tVar.NUMBERFORMAT = "%.3f"| & Defines the number format for printing. \\
		\lstinline|tVar.MATH_ENVIRONMENT = "align"| & Defines the environment used around equations. \\
		\lstinline|tVar.OUTPUT_MODE = "RES_EQ_N"| & Defines the outputmode (Section \ref*{sec:OUTPUT_PRINT}). \\
		\lstinline|tVar.EQUATION_NUMBERING = true| & Disables and enables numeration of equations. \\
		\lstinline|tVar.DECIMAL_SEPARATOR = "."| & Defines the decimal separator.\\
		\lstinline|tVar.MATRIX_COMMAND = "mathbf"| & Defines the style for matrices. \\
		\lstinline|tVar.MATRIX_EQUATION_AS_MATRIX = false| & Enables and disables output of a matrix as variable name or matrix with variable names\\
		\lstinline|tVar.UNIT_COMMAND = "\\si""| & Sets the macro for displaying units.\\
		\lstinline|tVar.VECTOR_COMMAND = "vec"| & Defines the style for vectors. \\\midrule
		
		\lstinline|tVar.CALC_PRECISION = 10| & Defines the how many decimal places are used for comparison.\\
		\lstinline|tVar.OUTPUT_DISABLED = false| & Disables the complete output.\\
		\lstinline|tVar.REMOVE_ZEROS = true| & Removes trailing zeros from a decimal number. \\
		\lstinline|tVar.REMOVE_DECIMAL_SEPARATOR = true| & In case \lstinline|tVar.REMOVE_DECIMAL_SEPARATOR| is true remove the decimal separator if all trailing zeros have been removed. Else show number with one decimal zero. (Only works if \lstinline|tVar.REMOVE_DECIMAL_SEPARATOR = true|)  \\\midrule
		
		\lstinline|tVar.DEBUG_MODE = "off"| & In case debugMode is set to "on" the equations are printed as \LaTeX code. \\
		\lstinline|tVar.DEBUG_LOG_COMMANDS_TO_FILE = false | & Enables logging of interpreted commands. Creates a file tVarLog.log\\
		\lstinline|tVar.OUTPUT_COLORED = false| & Prints all variables with value=nil red.\\
		\bottomrule\end{tabular}
\end{table}

\subsubsection{Details on automatic removal of trailing zeros}
The following example shows the difference between the global parameters
\lstinline|tVar.REMOVE_DECIMAL_SEPARATOR| and \lstinline|tVar.REMOVE_ZEROS|. 
If \lstinline|tVar.REMOVE_ZEROS| is disabled the number format from
\lstinline|tVar.NUMBERFORMAT| get's applied.
\begin{lstlisting}
\begin{tVar}
	#Default settings
	{
		a:=2.0
		b:=3.3
		c:=tVar.sqrt(a^2+b^2)
	}
	#Disable \\verb|tVar.REMOVE_DECIMAL_SEPARATOR|
	tVar.REMOVE_DECIMAL_SEPARATOR = false
	{
		a:=2.0
		b:=3.3
		c:=tVar.sqrt(a^2+b^2)
	}
	#Disable \\verb|tVar.REMOVE_ZEROS|
	tVar.REMOVE_ZEROS = false
	{
		a:=2.0
		b:=3.3
		c:=tVar.sqrt(a^2+b^2)
	}
\end{tVar}
\end{lstlisting}
\begin{tVarOUT}
\begin{tVar}
	#Default settings
{
	a:=2.0
	b:=3.3
	c:=tVar.sqrt(a^2+b^2)
}
	#Disable \\verb|tVar.REMOVE_DECIMAL_SEPARATOR|
	tVar.REMOVE_DECIMAL_SEPARATOR = false
{
	a:=2.0
	b:=3.3
	c:=tVar.sqrt(a^2+b^2)
}
	#Disable \\verb|tVar.REMOVE_ZEROS|
	tVar.REMOVE_ZEROS = false
{
	a:=2.0
	b:=3.3
	c:=tVar.sqrt(a^2+b^2)
}
	tVar.REMOVE_DECIMAL_SEPARATOR = true
	tVar.REMOVE_ZEROS = true
\end{tVar}
\end{tVarOUT}
\subsection{Plotting with Gnuplot}
Plotting in TexVar is support via Gnuplot. The code describing the figure is
generated in TexVar and is sent to Gnuplot which creates the graphics.

\subsubsection{Configuration}
For enabling plotting the path to the Gnuplot executable has to be set via a
global parameter. By default its set to \lstinline|tVar.GNUPLOT_LIBRARY
="gnuplot"|. In case you work on a windows system and want to specify the
absolute path to your Gnuplot install the command has to be
\lstinline|tVar.GNUPLOT_LIBRARY =[==["WINDOWSPATH"]==]|.

The Gnuplot terminal is by default \lstinline|tVar.GNUPLOT_TERMINAL =
"postscript eps enhanced color font 'Helvetica,12'"| and the file extension is
\lstinline|tVar.GNUPLOT_FILE_NAME_EXTENSION = "eps"|.
\subsubsection{Creating a Plot}
Every parameter set through \lstinline|[tPlot].*| is directly translated to a
gnuplot command. Actually tPlot is only aware of the following commands
\begin{table}[H]\centering
	\begin{tabular}{@{}p{5.6cm}p{8cm}@{}}\toprule
		\lstinline|tPlot:New(present[tPlot])| & Creates a new plot. Present is an optional parameter. If a value gets passed the configuration of the plot is used as template for the new plot. \\
		\lstinline|[tPlot].steps = 0.1| & Resolution for functions \\
		\lstinline|[tPlot].conf.size = "14cm,8cm"| & Size of the plot\\
		\lstinline|[tPlot].xrange = "[0,10]"| & Range of the x axis and min and max value for function creation.\\
		\lstinline|[tPlot]:add(f or {{1,2},{3,2}},"f(x)", "with line lt 1 lc 2")| & Adds a functions or points to the plot.\\
		\lstinline|[tPlot]:plot()| & Generates the plot and returns the includegraphics command. \\
		\bottomrule\end{tabular}
\end{table}
\begin{lstlisting}
\begin{tVar}
	f(x):=1/x
	
	plot1=tPlot:New()
	plot1.xlabel = "{/ Symbol e}_c"
	plot1.ylabel = "{/ Symbol s}_c"
	plot1.steps = 0.001
	
	plot1.xtics = "0.1"
	plot1.xrange = "[0:1]"
	plot1.yrange = "[0:16]"
	plot1.conf.size = "6cm ,6cm"
	plot1:add(f,"f(x)", "with line lt 1 lc 2")
	plot1:add({{0.2,6},{0.5,12}}, "pt", "with points lc 1")
	#\\begin{center}
		plot1:plot()
	#\\end{center}
\end{tVar}
\end{lstlisting}
\begin{tVarOUT}
\begin{tVar}
	tVar.GNUPLOT_LIBRARY=[==["C:\\Program Files (x86)\\gnuplot\\bin\\gnuplot.exe"]==]
	--tVar.GNUPLOT_LIBRARY="/usr/local/Cellar/gnuplot/5.0.1/bin/gnuplot"
		f(x):=1/x
		
		plot1=tPlot:New()
		plot1.xlabel = "{/ Symbol e}_c"
		plot1.ylabel = "{/ Symbol s}_c"
		plot1.steps = 0.001
		
		plot1.xtics = "0.2"
		plot1.xrange = "[0:1]"
		plot1.yrange = "[0:16]"
		plot1.conf.size = "5cm ,5cm"
		plot1:add(f,"f(x)", "with line lt 1 lc 2")
		plot1:add({{0.2,6},{0.5,12}}, "pt", "with points lc 1")
		#\\begin{center}
		plot1:plot()
		#\\end{center}
\end{tVar}
\end{tVarOUT}
Currently TexVar only supports 2D plots and functions with one attribute. If
you want to print a function with more than one attribute you can create a
helper function.
\begin{lstlisting}
\begin{tVar}
	-- use plot1 as template
	plot2=tPlot:New(plot1)
	
	f(x,z):=(1/x)^z
	-- helper functions with fixed z values
	f_h_1(x):=f(x,2);
	f_h_2(x):=f(x,1);
	
	plot2:add(f_h_1,"z=2","with line lt 1 lc 2")
	plot2:add(f_h_2,"z=1","with line lt 1 lc 3")
	
	#\\begin{center}
		plot2:plot()
	#\\end{center}
\end{tVar}
\end{lstlisting}
\begin{tVarOUT}
\begin{tVar}
	-- use plot1 as template
	plot2=tPlot:New(plot1)
	
	f(x,z):=(1/x)^z
	f_h_1(x):=f(x,2);
	f_h_2(x):=f(x,1);
	
	plot2:add(f_h_1,"z=2","with line lc 2")
	plot2:add(f_h_2,"z=1","with line lc rgb 'red' lw 2")
	
	#\\begin{center}
	plot2:plot()
	#\\end{center}
\end{tVar}
\end{tVarOUT}
\subsection{Mathematical Commands}
The following subsections are just a listing of currently implemented
mathematical functions.
\subsubsection{General}
These functions can be used with every tVar object.
\begin{itemize}[noitemsep]
	\item tVar.sqrt([tVar],[tVar])
	\item tVar.PI
	\item tVar.abs([tVar])
	\item tVar.acos([tVar]) 
	\item tVar.acosd([tVar]) 
	\item tVar.cos([tVar])
	\item tVar.cosd([tVar])
	\item tVar.cosh([tVar])
	\item tVar.asin([tVar])
	\item tVar.asind([tVar])
	\item tVar.sin([tVar])
	\item tVar.sind([tVar])
	\item tVar.sinh([tVar]) 
	\item tVar.atan([tVar]) 
	\item tVar.atand([tVar]) 
	\item tVar.tan([tVar])
	\item tVar.tand([tVar])
	\item tVar.tanh([tVar])
	\item tVar.ceil([tVar])
	\item tVar.floor([tVar]) 
	\item tVar.exp([tVar]) 
	\item tVar.ln([tVar])
	\item tVar.log10([tVar]) 
	\item tVar.atan2(X[tVar],Y[tVar])
	\item tVar.fact([tVar])
\end{itemize}
\subsubsection{Matrices and Vectors}
These functions can only be used with tMat or tVec objects.
\begin{itemize}[noitemsep]
	\item {[tMat]:T()}
	\item {[tMat]:Det()}
	\item {[tMat]:Inv()}
	\item {tVec.crossP([tVec],[tVec])}
\end{itemize}
\subsection{The Link Function}
If you want to use your own functions within TexVar, you can use the Link
function to link them to tVar functions.
This only applies to functions that don't use TexVar objects for calculation.
If you want to write a TexVar function see Section \ref{lbl:creatFun}.
The following code shows the implementation of the factorial function with the
link function.
\begin{lstlisting}
\begin{tVar}
	function mycalcFactorial(n)
		-- no tVar objects here just numbers
		if n<=1 then return 1 end
		return n*mycalcFactorial(n-1)
	end
	
	-- link it
	myfact = tVar.link(mycalcFactorial,"","!",nil,nil,nil,true)
	
	a:=10
	b:=myfact(a)
\end{tVar}
\end{lstlisting}
\begin{tVarOUT}
\begin{tVar}
	function mycalcFactorial(n)
	-- no tVar objects here just numbers
	if n<=1 then return 1 end
	return n*mycalcFactorial(n-1)
	end
	
	-- link it
	myfact = tVar.link(mycalcFactorial,"","!",nil,nil,nil,true)
	
	a:=10
	b:=myfact(a)
\end{tVar}
\end{tVarOUT}
The link function has the following attributes:
\begin{itemize}
	\item \verb|function|: The function that should be transformed to a TexVar function.
	\item \verb|texBefore|: Text which is added before the return value.
	\item \verb|texAfter|: Text which is added after the return value.
	\item \verb|returntype (optional)|: Defines the type of the return objecte, possible values are tVar, tMat, tVec
	\item \verb|inputUnit (optional)|: Required unit for input attributes. 
	\item \verb|outputUnit (optional)|: Unit of returned function.
	\item \verb|pipeUnit (optional)|: If true, the input and output unit is derived from the first passed argument.
\end{itemize}

All Lua math functions are implemented this way. 

For example the Lua function \lstinline|math.atan2|:
\begin{lstlisting}
--- calculates inverse tangens with with appr. quadrant
-- 
-- @param opposite (tVar,number) values
-- @param adjacent (tVar,number) values
-- @return (tVar) 
tVar.link(math.atan2,"\\text{atan2}\\left(","\\right)",nil,tVar.units("m"),tVar.units("rad"),false)
\end{lstlisting}
\begin{tVarOUT}
\begin{tVar}
	{
	a:=3:setUnit("m")
	b:=4:setUnit("m")
	c:=tVar.atan2(a,b)
	}
\end{tVar}
\end{tVarOUT}
%\section{Unit based Calculations}\label{sec:unit-based-calculations}
%This section describes the functionality and the concept of the unit based calculations of TexVar introduced in version $1.5.18$.
%\subsection{The Concept}
%Basically every unit is defined as a tree. Some units like meter only consist of one leaf, others like Mega Pascal get pretty complex:
%\begin{tVarOUT}
%	\begin{minipage}[t]{0.45\textwidth}
%\begin{tVar}
%	tVar.units.unitset["m"]:printStackTex()
%\end{tVar}
%	\end{minipage}
%	\begin{minipage}[t]{0.45\textwidth}
%\begin{tVar}
%	tVar.units.unitset["MPa"]:printStackTex()
%\end{tVar}
%	\end{minipage}
%\end{tVarOUT}
%
%Units with only one leaf are called base units. Every unit can be reduced to a tree of base units which makes it comparable to other units. If you set a unit for a variable in TexVar, the value is always converted to base units. For the output TexVar uses the preferred unit you defined with \verb|:setUnit()|. The following example show the difference between the real output and the stored values.
%\begin{tVarOUT}
%\begin{tVar}
%	a:=10:setUnit("cm")
%	# The value of a is: 
%	tex.print(a.val .. "\\\\")
%	# The unit of a is: 
%	tex.print("\\si{" .. a.unit:toString() .. "}\\\\")
%	# The preferred unit of a is: 
%	tex.print("\\si{" .. a.prefUnit:toString() .. "}\\\\")
%\end{tVar}
%\end{tVarOUT}
%\subsection{Special Functions}
%The following list contains all functions which are used for units:
%\begin{itemize}
%	\item \verb|[tVar]:setUnit([string])|: tries to interpret the given string as unit. For example \si{m} is known as meter and \si{kN/m^2} is separated into \si{kN} divided by \si{m^2}, which are both known units \footnote{Powers are not recognized as such, which means that \si{m^2} is a unit, defined as \si{m} times \si{m}}. For those strings, describing units, you can use:
%	\begin{itemize}
%		\item Round brackets for enclosure,
%		\item . for multiplication and
%		\item / for division.
%	\end{itemize}
%	When you define a unit the value is multiplied with a factor to convert it to a base unit.
%	\item \verb|[tVar]:clearUnit([string, optional])|: removes all unit information. If you define a unit with the string attribute the value is initially converted to this unit and the unit-information is removed afterwards. This comes in handy if you use empirical formulas which are not bound to units, for a demonstration have a look at the Example \ref{ssec:windchill}.
%	\item \verb|[tVar]:setUText([string])|: Overrides the unit output, in case you want to have a specific format.
%\end{itemize}
%\begin{lstlisting}
%\begin{tVar}
%	{
%		a:=10:setUnit("cm")
%		b:=1.2:setUnit("mm")
%		c:=a+b
%	}
%	# If you try to convert something to another unit TexVar gives you a possible solution containing the chosen unit.\\\\
%	# Reasonable conversion:
%	tVar.outRES(c:setUnit("m"))
%	# Unreasonable conversion:
%	tVar.outRES(c:setUnit("K.cm"))
%	# It is always possible to override the output text of a unit:
%	tVar.outRES(c:setUText("\\newton\\per\\meter"))
%\end{tVar}
%\end{lstlisting}
%\begin{tVarOUT}
%\begin{tVar}
%{
%	a:=10:setUnit("cm")
%	b:=1.2:setUnit("mm")
%	c:=a+b
%}
%# If you try to convert something to another unit TexVar gives you a possible solution containing the chosen unit.\\\\
%# Reasonable conversion:
%tVar.outRES(c:setUnit("m"))
%# Unreasonable conversion:
%tVar.outRES(c:setUnit("K.cm"))
%# It is always possible to override the output text of a unit:
%tVar.outRES(c:setUText("\\newton\\per\\meter"))
%\end{tVar}
%\end{tVarOUT}
%\subsection{List of known Units \label{ssec:list_of_known_units}}
%The following list is a dump of the global variable \verb|tVar.units.unitset| which holds all registered units.
%%\begin{lstlisting}
%%\begin{tVar}
%%	local outp = {}
%%	for uni,_ in pairs(tVar.units.unitset) do
%%		table.insert(outp,"\\si{" .. uni .. "}")
%%	end
%%	table.sort(outp)
%%	tex.print(table.concat(outp,", "))
%%\end{tVar}
%%\end{lstlisting}
%\begin{tVarOUT}
%\begin{tVar}
%	local outp = {}
%	for i,v in pairs(tVar.units.unitset) do
%	table.insert(outp,"\\si{" .. i .. "}")
%	end
%	table.sort(outp)
%	tex.print(table.concat(outp,", "))
%\end{tVar}
%\end{tVarOUT}
%\subsection{Defining new Units}
%As we already know, a unit consist of other units. To create a new unit you have to link other units with factors and operations together. Possible operations are:
%\begin{itemize}
%	\item \verb|[units]:mul(factor,[units] optional)|
%	\item \verb|[units]:div(factor,[units] optional)|
%	\item \verb|[units]:root(scale)| 
%	\item \verb|[units]:add(scale)|\footnote{The add command is currently not supported by the unit library. In favour of future development all required functions have already been implemented.}
%\end{itemize}
%Afterwards you have to register your unit to the global unit stack and assign a unique name with \verb|:addUnit([string])|. In the following example the unit \si{Ar} is defined by $100 \cdot \si{m^2}$, afterwards assigned to a variable, which is then converted to \si{cm^2}.
%\begin{lstlisting}
%\begin{tVar}
%tVar.units():mul(100,tVar.units("m^2")):addUnit("Ar")
%A:=12:setUnit("Ar")
%A:=A:setUnit("cm^2"):outRES()
%\end{tVar}
%\end{lstlisting}
%\begin{tVarOUT}
%\begin{tVar}
%	tVar.units():mul(100,tVar.units("m^2")):addUnit("Ar")
%	A:=12:setUnit("Ar")
%	A:=A:setUnit("cm^2"):outRES()
%\end{tVar}
%\end{tVarOUT}
%All units are registered in a file called \verb|unit_definitions.lua|. It's found in the tVar folder inside lib. At the moment only a small amount of units is implemented.
\section{Examples}
\subsection{U-Value}
This is a very simple example using only the basic functionality of TexVar.
\begin{lstlisting}
Calculating the U-Value for an element with two layers.\\
\begin{tVar}
	#\\noindent Resistance of surface
	{plain
		R_se:= 0.3:setUnit("m^2.K/W")
		#,~
		R_si:= 0.13:setUnit("m^2.K/W")
	}
	#Parameters for elements
	{
		d_1 := 0.20:setUnit("m")
		\\lambda_1 := 0.035:setUnit("W/(m.K)")
		d_2 := 0.10:setUnit("m")
		\\lambda_2 := 0.5:setUnit("W/(m.K)")
	}
	#Calculate thermal resistance
	R := (R_se + d_1/lambda_1 + d_2/lambda_2 + R_si):setUnit("m^2.K/W")
	#Calculate U-Value
	U:=(1/R):setUnit("W/(m^2.K)")
\end{tVar}
\end{lstlisting}
\begin{tVarOUT}
Calculating the U-Value for an element with two layers.\\
\begin{tVar}
	#\\noindent Resistance of surface
	{plain
		R_se:= 0.3:setUnit("m^2.K/W")
		#,~
		R_si:= 0.13:setUnit("m^2.K/W")
	}
	#Parameters for elements
	{
		d_1 := 0.20:setUnit("m")
		\\lambda_1 := 0.035:setUnit("W/(m.K)")
		d_2 := 0.10:setUnit("m")
		\\lambda_2 := 0.5:setUnit("W/(m.K)")
	}
	#Calculate thermal resistance
	R := (R_se + d_1/lambda_1 + d_2/lambda_2 + R_si):setUnit("m^2.K/W")
	#Calculate U-Value
	U:=(1/R):setUnit("W/(m^2.K)")
\end{tVar}
\end{tVarOUT}
\subsection{Rotating a Vector}
This example shows the usage of the parameter \lstinline|tMat.eqTexAsMatrix| as
mentioned in Section \ref{sec:globparam}.
\begin{lstlisting}
\begin{tVar}
tMat.MATRIX_EQUATION_AS_MATRIX = true

#Rotation angle in degree
\\theta:=45:setUnit("°")

#Rotation matrix in R2
{
A := {{tVar.cos(theta),-tVar.sin(theta)},{tVar.sin(theta),tVar.cos(theta)}}:outRES_EQ()
e_x:={1,0}
}


tMat.MATRIX_EQUATION_AS_MATRIX = false
e:=(A*e_x)
\end{tVar}
\end{lstlisting}
\begin{tVarOUT}
\begin{tVar}
tMat.MATRIX_EQUATION_AS_MATRIX = true
#Rotation angle in degree
\\theta:=45:setUnit("°")

#Rotation matrix in R2
{
A := {{tVar.cos(theta),-tVar.sin(theta)},{tVar.sin(theta),tVar.cos(theta)}}:outRES_EQ()
e_x:={1,0}
}

tMat.MATRIX_EQUATION_AS_MATRIX = false
e:=(A*e_x)
\end{tVar}
\end{tVarOUT}
\subsection{Vector Calculations - Custom Function}
This example shows how to create a custom TexVar function for calculating the
angle between two vectors. The function has an extra parameter for disabling
printing.
\begin{lstlisting}
\begin{tVar}
	
tVar.OUTPUT_MODE = "RES_EQ"	
function angleBetweenVectors(a,b,disablePrinting)
	if disablePrinting then
		tVar.OUTPUT_DISABLED = true
	end
	
	#Calculate the length of the vectors
	
	len_a:=tVar.sqrt(a*a)
	len_b:=tVar.sqrt(b*b)
	
	#Normalize the vectors
	
	a_n:=(a/len_a)
	b_n:=(b/len_b)
	
	\\alpha:=tVar.acosd(a_n*b_n):setUnit("°")
	
	if disablePrinting then
		tVar.OUTPUT_DISABLED = false
	end
	
	return alpha
end
	
#With output
{
	v_1:={1,0.4,0.5}
	v_2:={0.3,1,0}
}
	
\\alpha_1:=angleBetweenVectors(v_1,v_2)
	
#Without output
{
	v_3:={4,0.2,5}
	v_4:={9.3,8,1}
}
\\alpha_2:=angleBetweenVectors(v_3,v_4,true)
\end{tVar}
\end{lstlisting}
\begin{tVarOUT}
\begin{tVar}
	
tVar.OUTPUT_MODE = "RES_EQ"	
function angleBetweenVectors(a,b,disablePrinting)
	if disablePrinting then
		tVar.OUTPUT_DISABLED = true
	end

	#Calculate the length of the vectors

	len_a:=tVar.sqrt(a*a)
	len_b:=tVar.sqrt(b*b)

	#Normalize the vectors

	a_n:=(a/len_a)
	b_n:=(b/len_b)

	\\alpha:=tVar.acos(a_n*b_n):setUnit("°")

	if disablePrinting then
		tVar.OUTPUT_DISABLED = false
	end

	return alpha
end

#With output
{
v_1:={1,0.4,0.5}
v_2:={0.3,1,0}
}

\\alpha_1:=angleBetweenVectors(v_1,v_2)

#Without output
{
	v_3:={4,0.2,5}
	v_4:={9.3,8,1}
}
\\alpha_2:=angleBetweenVectors(v_3,v_4,true)

\end{tVar}
\end{tVarOUT}

%\subsection{Trigonometric Functions}
%Trigonometric functions are capable of converting angles automatically into radiants.
%\begin{lstlisting}
%\begin{tVar}
%#Rectangular triangle defined by the following values:
%{
%a:=3:setUnit("cm")
%b:=40:setUnit("mm")
%}
%c:=tVar.sqrt(a^2+b^2):setUnit("m")
%{
%\\alpha:=tVar.asin(b/c):setUnit("°")
%\\beta:=tVar.asin(a/c):setUnit("°")
%}
%tVar.outRES(tVar:New(180):setUnit("°")-alpha-beta)
%a_2:=(c*tVar.cos(alpha)):setUnit("m")
%if a_2 == a then
%# Comparing Units worked!
%else
%# Comparing Units failed!
%end
%\end{tVar}
%\end{lstlisting}
%\begin{tVarOUT}
%\begin{tVar}
%	#Rectangular triangle defined by the following values:
%	{
%		a:=3:setUnit("cm")
%		b:=40:setUnit("mm")
%	}
%	c:=tVar.sqrt(a^2+b^2):setUnit("m")
%	{
%		\\alpha:=tVar.asin(b/c):setUnit("°")
%		\\beta:=tVar.asin(a/c):setUnit("°")
%	}
%	tVar.outRES(tVar:New(180):setUnit("°")-alpha-beta)
%	a_2:=(c*tVar.cos(alpha)):setUnit("m")
%	if a_2 == a then
%	# Comparing Units worked!
%	else
%	# Comparing Units failed!
%	end
%\end{tVar}
%\end{tVarOUT}
%\subsection{Windchill \label{ssec:windchill}}
%The purpose of this example is to demonstrate the usage of the \verb|:clearUnit()| function when doing calculations based on empiric formulas. 
%\begin{lstlisting}
%\begin{tVar}
%# Windvelocity:
%V:=3:setUnit("m/s")
%# Air temperature:
%T_a:=10:setUnit("°C")
%# Windchill (without removing units, throws error):
%T_wc:=(13.12 + 0.6215*T_a - 11.37*V^0.16+0.3965*T_a*V^0.16):CRLF_EQ("=")
%# Remove units and transform %%V%% to km/h
%V:=V:clearUnit("km/h");
%T_a:=T_a:clearUnit();
%T_wc:=(13.12 + 0.6215*T_a - 11.37*V^0.16+0.3965*T_a*V^0.16):setUnit("°C")
%\end{tVar}
%\end{lstlisting}
%\begin{tVarOUT}
%\begin{tVar}
%	# Windvelocity:
%	V:=3:setUnit("m/s")
%	# Air temperature:
%	T_a:=10:setUnit("°C")
%	# Windchill (without removing units, throws error):
%	T_wc:=(13.12 + 0.6215*T_a - 11.37*V^0.16+0.3965*T_a*V^0.16):CRLF_EQ("=")
%\end{tVar}
%\begin{tVar}
%	# Remove units and transform %%V%% to km/h
%	V:=V:clearUnit("km/h");
%	T_a:=T_a:clearUnit();
%	T_wc:=(13.12 + 0.6215*T_a - 11.37*V^0.16+0.3965*T_a*V^0.16):CRLF_EQ("="):setUnit("°C")
%\end{tVar}
%\end{tVarOUT}
\end{document}
